 \begin{DoxyAuthor}{Authors}
Cosmin Mihai 
\end{DoxyAuthor}
\begin{DoxyVersion}{Version}
1.\-0\par
\par
 
\end{DoxyVersion}
\begin{DoxyDate}{Date}
2012 
\end{DoxyDate}
\begin{DoxyCopyright}{Copyright}
G\-N\-U Public License.
\end{DoxyCopyright}

\begin{DoxyItemize}
\item email\-: \href{mailto:yonutix@yahoo.com}{\tt yonutix@yahoo.\-com}\par

\item website\-: \href{http://www.freya.zzl.org}{\tt Freya}
\end{DoxyItemize}

\section*{Fisierele de configurare\-:}

{\bfseries Fisierul de configurare pentru obiectele statice\-:}\par
 Proiectul trebuie sa contina in radacina un fisier \char`\"{}static\-\_\-obj.\-txt\char`\"{} cu urmatorul format\-:\par
 name=\mbox{[}nume obiect\mbox{]}\par
 filename=\mbox{[}adresa obiect in format binar(vezi formatul binar al obiectelor)\mbox{]}\par
 texture=\mbox{[}adresa imaginii care va fi folosita ca textura in format T\-G\-A 8\-B uncompressed\mbox{]}\par
 poz\-X=\mbox{[}pozitia pe axa ox\mbox{]}\par
 poz\-Y=\mbox{[}pozitia pe axa oy\mbox{]}\par
 poz\-Z=\mbox{[}pozitia pe axa oz\mbox{]}\par
 angle=\mbox{[}unghiul rotatiei\mbox{]}\par
 rot\-X=\mbox{[}componenta pe ox a vectorului rotatie\mbox{]}\par
 rot\-Y=\mbox{[}componenta pe oy a vectorului rotatie\mbox{]}\par
 rot\-Z=\mbox{[}componenta pe oz a vectorului rotatie\mbox{]}\par
 scale\-X=\mbox{[}componenta pe ox a scalarii\mbox{]}\par
 scale\-Y=\mbox{[}componenta pe oy a scalarii\mbox{]}\par
 scale\-Z=\mbox{[}componenta pe oz a scalarii\mbox{]}\par
 //scale\-X, scale\-Y, scale\-Z nu sunt folosite in verisunea 1.\-0\par
 boundingbox=\mbox{[}Adresa bounding-\/box-\/ului in format bb -\/ vezi format bb\mbox{]}\par
 {\bfseries Fisierul de configurare pentru highmap}\par
 Se numeste \char`\"{}high\-Map.\-txt\char`\"{} si contine\-:\par
 lines=\mbox{[}numarul de vertecsi de-\/a lungul unei coloane\mbox{]}\par
 columns=\mbox{[}numarul de vertecsi de-\/a lungul unei linii\mbox{]}\par
 line\-Step=\mbox{[}distanta dintre 2 vertecsi de pe linii consecutive\mbox{]}\par
 column\-Step=\mbox{[}distanta dintre 2 coloane de vertecsi consecutive\mbox{]}\par
 \section*{Formate fisiere\-:}

{\bfseries .bin}\par
 Fisier binar\-:\par
 $|$\-Numar vertecsi sizeof(int)$|$\textbackslash{} v(veretex.\-x vertex.\-y vertex.\-z) sizeof(float)$\ast$ nr vertecsi $\ast$ 3 \textbackslash{}\par
 $|$\-Numar normale sizeof(int)$|$\textbackslash{} v(normala.\-x normala.\-y normala.\-z) sizeof(float)$\ast$ nr normale $\ast$ 3 \textbackslash{}\par
 $|$\-Numar coordonate textura sizeof(int)$|$\textbackslash{} v(map.\-x map.\-y) sizeof(float)$\ast$ nr normale $\ast$ 2 \textbackslash{}\par
 $|$\-Numar fete sizeof(int)$|$\textbackslash{} numar fete $\ast$ sizeof(\-Face)\textbackslash{}\par
 {\bfseries .bb}\par
 Reprezinta un set de bounding-\/box-\/uri\par
 In versiunea 1.\-0 exista 2 tipuri de bounding-\/box-\/uri\-:\par

\begin{DoxyItemize}
\item 0 Paralelipiped care este reprezentat de 6 float-\/uri(min\-Point, max\-Point)\par

\item 1 Sfera care este reprezentata de 4 float-\/uri pozitia, raza\par

\item Pe prima linie se afla numarul de bounding-\/box-\/uri\par
 pe urmaroaterele se afla informatiile despre fiecare bounding-\/box sfera sau paralelipiped\par
 
\end{DoxyItemize}